\documentclass{article}
\title{\textbf{Mass–energy equivalence}}
\date{June 2022}
\author{Jaydeep raut MM21B032}

\usepackage{graphicx}


\begin{document}
  \maketitle
\large
\paragraph{}In physics, \textbf{mass-energy equivalence} is the relationship between mass and energy in a system's rest frame, where the two values differ only by a constant and the units of measurement. The principle is described by the physicist Albert Einstein's famous formula: 

\vspace{1cm}


\boldmath
\begin{equation}
  E=mc^{2}
\end{equation}

\vspace{1cm}

the formula implies that a small amount of rest mass corresponds to an enormous amount of energy, which is independent of the composition of the matter. 

\vspace{1cm}

\begin{tabular}{|c|l|}
    \hline
    $E$ & Energy \\
    $m$ & Mass\\
    $c$ & The speed of light (approximately 300000 km/s)  \\
    \hline
\end{tabular}


\end{document}
